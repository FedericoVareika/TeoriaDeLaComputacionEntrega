\subsubsection*{Reducci\'on 3-SAT $\le_{p}$ Subset Sum}

Cabe aclarar que esta reducci\'on no fue ideada por nosotros, sino que tomamos 
como referencia la reducci\'on mostrada en \cite{pcmi2007lectureB07}.

Para realizar esta reducci\'on tenemos que moldar el problema \textbf{3-SAT} para 
que se pueda reducir de manera trivial a \textbf{Subset Sum}.

Tomamos el problema \textbf{3-SAT} como un conjunto de variables $x_1, ..., x_n$
y un conjunto de clausulas $c_1, ..., c_r$. 
Para cada variable $x_i$, se puede decir que tenemos dos opciones, que esta se 
evalue a \texttt{True} o que se evalue a \texttt{False}, llamaremos a estas 
opciones como $a_i$ y $b_i$ respectivamente. 
Si se toma la opcion $b_i$, entonces podemos concluir que cada clausula que 
tenga a la variable $x_i$ negada se evaluar\'a a \texttt{True}. 

Si ahora tomamos estas opciones como numeros decimales de manera que: 

\[
  a_i = b_i = 10^{n-i}
\]

Entonces podemos formar la siguiente tabla tomando $n=3$: 

\begin{center}
\begin{tabular}{ c|c c c }
  & $x_1$ & $x_2$ & $x_3$ \\
  \hline 
  $a_1$ & 1 & 0 & 0 \\
  $b_1$ & 1 & 0 & 0 \\
  $a_2$ & 0 & 1 & 0 \\
  $b_2$ & 0 & 1 & 0 \\
  $a_3$ & 0 & 0 & 1 \\
  $b_3$ & 0 & 0 & 1 \\
\end{tabular} 
\end{center}

\newpage 
  
En \textbf{3-SAT} solo podemos tomar una opcion por variable, y debemos definir 
todas las variables. Dado esto, si sumamos todas las opciones elegidas, y 
llamamos esta suma $k$: 

\begin{table}[h!]
  \centering
  \begin{tabular}{ c|c c c }
    & $x_1$ & $x_2$ & $x_3$ \\
    \hline 
    $a_1$ & 1 & 0 & 0 \\
    \rowcolor{green!30}
    $b_1$ & 1 & 0 & 0 \\
    \rowcolor{green!30}
    $a_2$ & 0 & 1 & 0 \\
    $b_2$ & 0 & 1 & 0 \\
    $a_3$ & 0 & 0 & 1 \\
    \rowcolor{green!30}
    $b_3$ & 0 & 0 & 1 \\
    \hline
    $k$ & 1 & 1 & 1 \\
  \end{tabular} 
  \caption*{Se eligen las opciones en verde como ejemplo.}
\end{table}

Teniendo esto y usando el hecho que si elegimos $a_i$ entonces cada clausula 
que contenga $x_i$ no negada se va a evaluar a \texttt{True}, podemos extender esta 
tabla. 
Para el ejemplo anterior tomamos la siguiente expresion: 

\[
  \underbrace{(x_1 \lor \neg x_2 \lor x_3)}_{c_1} 
  \land 
  \underbrace{(\neg x_1 \lor \neg x_2 \lor \neg x_3)}_{c_2} 
\]

Y derivamos la tabla: 

\begin{table}[h!]
  \centering
  \begin{tabular}{ c|c c c | c c }
    & $x_1$ & $x_2$ & $x_3$ & $c_1$ & $c_2$ \\
    \hline 
    $a_1$ & 1 & 0 & 0 & 1 & 0 \\
    \rowcolor{green!30}
    $b_1$ & 1 & 0 & 0 & 0 & 1 \\
    \rowcolor{green!30}
    $a_2$ & 0 & 1 & 0 & 0 & 0 \\
    $b_2$ & 0 & 1 & 0 & 1 & 1 \\
    $a_3$ & 0 & 0 & 1 & 1 & 0 \\
    \rowcolor{green!30}
    $b_3$ & 0 & 0 & 1 & 0 & 1 \\
    \hline
    $k$ & 1 & 1 & 1 & 0 & 3 \\
  \end{tabular} 
\end{table}

Se puede ver en esta tabla como indicamos con un $1$ las clausulas que se 
evaluan a \texttt{True} al elegir la opcion en esa fila, e indicamos con $0$ 
el resto.  
Tambien extendimos $k$ para incluir la suma de las opciones marcadas.

\newpage

Podemos saber que se cumple \textbf{3-SAT} con las opciones elegidas sii se 
elige solo una opcion de cada variable \ref{eq:k1s} y todas las clausulas son
evaluadas a \texttt{True} al menos una vez \ref{eq:kge1s}. 

Estas reglas se ven cumplidas en los digitos de $k$:

\begin{numcases}{digitoK(i)=}
  1 & si $i \in [1, n]$
  \label{eq:k1s}\\
  v & si $i \in [n+1, n+r], \quad v \geq 1$
  \label{eq:kge1s}
\end{numcases}

Para el ejemplo anterior, es claro que una solucion es 
$x_1=\texttt{True}$, 
$x_2=\texttt{False}$, 
$x_2=\texttt{True}$.
Por lo tanto, deberiamos poder comprobarlo con estas nuevas reglas: 

\begin{table}[h!]
  \centering
  \begin{tabular}{ c|c c c | c c }
    & $x_1$ & $x_2$ & $x_3$ & $c_1$ & $c_2$ \\
    \hline 
    \rowcolor{green!30}
    $a_1$ & 1 & 0 & 0 & 1 & 0 \\
    $b_1$ & 1 & 0 & 0 & 0 & 1 \\
    $a_2$ & 0 & 1 & 0 & 0 & 0 \\
    \rowcolor{green!30}
    $b_2$ & 0 & 1 & 0 & 1 & 1 \\
    \rowcolor{green!30}
    $a_3$ & 0 & 0 & 1 & 1 & 0 \\
    $b_3$ & 0 & 0 & 1 & 0 & 1 \\
    \hline
    $k$ & 1 & 1 & 1 & 3 & 1 \\
  \end{tabular} 
\end{table}

\[
  k = 11131 \Rightarrow k[i] = digitoK(i) \quad \forall i \in [1, n + r]
\]

Podemos ver como nos estamos acercando a \textbf{Subset Sum}: 
\begin{itemize}
  \item Tenemos una entrada de numeros ($a_i$ y $b_i$).
  \item Sumamos un subconjunto de estos numeros.
  \item Verificamos que esta suma cumple ciertas reglas.
\end{itemize}
\label{itm:reglasSS}

El ultimo paso tendria que ser "verificamos que esta suma es igual a un numero 
$k'$. Para esto, debemos modificar las reglas originales: 

\begin{numcases}{digitoK'(i)=}
  1 & si $i \in [1, n]$
  \label{eq:k1s'}\\
  3 & si $i \in [n+1, n+r]$
  \label{eq:k3s}
\end{numcases}

\newpage

Y agregamos nuevos elementos al conjunto de numeros. Estos van a ser dos por 
clausula, de tal manera que se podra 'rellenar' de cierta forma el digito de 
esta clausula. Llamaremos a estos numeros $s_i$ y $t_i$.
Podemos ver estos nuevos numeros en la siguiente tabla: 

\begin{table}[h!]
  \centering
  \begin{tabular}{ c|c c c | c c }
    & $x_1$ & $x_2$ & $x_3$ & $c_1$ & $c_2$ \\
    \hline 
    \rowcolor{green!30}
    $a_1$ & 1 & 0 & 0 & 1 & 0 \\
    $b_1$ & 1 & 0 & 0 & 0 & 1 \\
    $a_2$ & 0 & 1 & 0 & 0 & 0 \\
    \rowcolor{green!30}
    $b_2$ & 0 & 1 & 0 & 1 & 1 \\
    \rowcolor{green!30}
    $a_3$ & 0 & 0 & 1 & 1 & 0 \\
    $b_3$ & 0 & 0 & 1 & 0 & 1 \\
    \hline
    $s_1$ & 0 & 0 & 0 & 1 & 0 \\
    $t_1$ & 0 & 0 & 0 & 1 & 0 \\
    $s_2$ & 0 & 0 & 0 & 0 & 1 \\
    $t_2$ & 0 & 0 & 0 & 0 & 1 \\
    \hline
    $k$ & 1 & 1 & 1 & 3 & 3 \\
  \end{tabular} 
\end{table}

Ahora podemos elegir los nuevos numeros para rellenar el ultimo digito de $k$:

\begin{table}[h!]
  \centering
  \begin{tabular}{ c|c c c | c c }
    & $x_1$ & $x_2$ & $x_3$ & $c_1$ & $c_2$ \\
    \hline 
    \rowcolor{green!30}
    $a_1$ & 1 & 0 & 0 & 1 & 0 \\
    $b_1$ & 1 & 0 & 0 & 0 & 1 \\
    $a_2$ & 0 & 1 & 0 & 0 & 0 \\
    \rowcolor{green!30}
    $b_2$ & 0 & 1 & 0 & 1 & 1 \\
    \rowcolor{green!30}
    $a_3$ & 0 & 0 & 1 & 1 & 0 \\
    $b_3$ & 0 & 0 & 1 & 0 & 1 \\
    \hline
    $s_1$ & 0 & 0 & 0 & 1 & 0 \\
    $t_1$ & 0 & 0 & 0 & 1 & 0 \\
    \rowcolor{green!30}
    $s_2$ & 0 & 0 & 0 & 0 & 1 \\
    \rowcolor{green!30}      
    $t_2$ & 0 & 0 & 0 & 0 & 1 \\
    \hline
    $k$ & 1 & 1 & 1 & 3 & 3 \\
  \end{tabular} 
\end{table}

Y ahora: 

\[
    k = 11133 \Rightarrow k[i] = digitoK'(i) \quad \forall i \in [1, n + r]
\]

PPor lo tanto, ya podemos cambiar la tercer regla de \ref{itm:reglasSS} por:
\begin{itemize}
  \item Verificamos que la suma es igual a 
    $\underbrace{1...1}_{n}\underbrace{3...3}_{r}$
\end{itemize}

Con esto, la reduccion esta completa. 
Se puede ver como crear el conjunto de numeros $a_i$ y $b_i$ es recorrer la lista 
de clausulas por cada variable ($O(n*r)$), y crear el conjunto de numeros 
$s_i$ y $t_i$ es recorrer la lista de clausulas ($O(r)$), por lo tanto es 
una reduccion polinomial.

Formalmente, el dominio de \textbf{3-SAT} ([Variables], [Clausulas]) se reduce 
a el dominio de \textbf{Subset Sum} ([Numeros], $k$)
